\documentclass[lang=cn,newtx,12pt,scheme=chinese]{elegantbook}



\begin{document}

LatexEdition文件夹中的内容是Latex书籍模板\lstinline{ElegantBook}原始压缩包中的标准模板内容,我想利用这个模板来书写我自己的
书籍/笔记,因此我创建了一个与LatexEdition并列的文件夹:\lstinline{MyElegantBook},准备一点一点学习ElegantBook书籍模板中的惯用法。

\textbf{如果你想使用ElegantBook这个模板,那么MyElegantBook这个目录下面一定要有elegantbook.cls}。

\href{https://blog.csdn.net/qq_44935032/article/details/130721017}{Ubuntu Snap Store无法升级的三种解决办法}(CSDN),
我在华硕电脑上遇到了Snap Store无法升级的情况:系统提示我有两个软件可以升级,我点击升级按钮提示错误。错误如下:

\begin{quotation}
    Unable to install updates: (null): cannot refresh "snap-store": snap "snap-store" has running apps (ubuntu-software), 
    pids: 2724.
\end{quotation}

解决办法就是先用\textit{ps -p 2724}查看一下这个进程是干什么的,然后用\textit{kill 2724}杀死这个进程,最后\textit{sudo snap refresh snap-store}
对snap-store进行更新。

在不知道进程号2724时用\textit{ps aux|grep snap-store}来查看snap-store程序对应的进程号。

Linux用户与用户组的设计,是为了权限管理吗?一个Shell脚本也只是一个普通文件,只不过文件内容是一堆命令。写成一个Shell脚本后需要给这个脚本文件赋权才能执行。
常见的对Shell脚本赋权的命令:\textit{chmod a+x script.sh}. 该命令赋予所有用户script.sh的可执行权限。







\end{document}

